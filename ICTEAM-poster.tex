% This template was written by Benoît Legat and is inspired from https://fr.overleaf.com/latex/examples/quadratic-function/hjbvztxdrvwf

\documentclass[final]{beamer}

\usepackage[scale=1.24]{beamerposter}
\usetheme{ICTEAMposter} % `ICTEAMposter` theme of `beamerposter` package

%-----------------------------------------------------------
% Define the column widths and overall poster size
% To set effective sepwid, onecolwid and twocolwid values, first choose how many columns you want and how much separation you want between columns
% In this template, the separation width chosen is 0.024 of the paper width and a 4-column layout
% onecolwid should therefore be (1-(# of columns+1)*sepwid)/# of columns e.g. (1-(4+1)*0.024)/4 = 0.22
% Set twocolwid to be (2*onecolwid)+sepwid = 0.464
% Set threecolwid to be (3*onecolwid)+2*sepwid = 0.708
\newlength{\sepwid}
\newlength{\onecolwid}
\newlength{\twocolwid}
\newlength{\threecolwid}
\setlength{\paperwidth}{36in} % A0 width: 46.8in
\setlength{\paperheight}{48in} % A0 height: 33.1in
\setlength{\sepwid}{0.024\paperwidth} % Separation width (white space) between columns
\setlength{\onecolwid}{0.3\paperwidth} % Width of one column
%\setlength{\twocolwid}{0.464\paperwidth} % Width of two columns
\setlength{\threecolwid}{0.708\paperwidth} % Width of three columns
\setlength{\topmargin}{-0.5in} % Reduce the top margin size
%-----------------------------------------------------------

\usepackage{booktabs} % Top and bottom rules for tables

\usepackage{biblatex}
\bibliography{biblio.bib}

\title{ICTEAM poster template}

\author{John Doe$^*$, Jean Dupont$^\dagger$}

\institute{$^*$ ICTEAM (UCLouvain), $^\dagger$ LIDS (MIT)}

%----------------------------------------------------------------------------------------

% https://tex.stackexchange.com/questions/426088/texlive-pretest-2018-beamer-and-subfig-collide
\makeatletter
\let\@@magyar@captionfix\relax
\makeatother

\begin{document}

\addtobeamertemplate{block end}{}{\vspace*{2ex}} % White space under blocks
\addtobeamertemplate{block alerted end}{}{\vspace*{2ex}} % White space under highlighted (alert) blocks

\setlength{\belowcaptionskip}{2ex} % White space under figures
\setlength\belowdisplayshortskip{2ex} % White space under equations

\begin{frame}[t,fragile] % The whole poster is enclosed in one beamer frame

\begin{columns}[t] % The whole poster consists of three major columns, the second of which is split into two columns twice - the [t] option aligns each column's content to the top

\begin{column}{\sepwid}\end{column} % Empty spacer column

\begin{column}{\onecolwid} % The first column
  \begin{block}{ICTEAM institute}
  \begin{alertblock}{What does ICTEAM stands for ?}
    Information and Communication Technologies, Electronics and Applied Mathematics
  \end{alertblock}

  The Institute is currently home for more than 40 professors and more than 200 researchers. These researchers carry out both basic and applied research in key fields of information and communication technologies, electronics, computer science and applied mathematics. Have a look at our publications : they show the breadth and the depth of our research.

  \end{block}

\end{column}

\begin{column}{\sepwid}\end{column} % Empty spacer column

\begin{column}{\onecolwid} % The second column
  \begin{block}{ICTEAM history}
    For a few decades, the research at UCLouvain in computer science, electrical engineering and applied mathematics was organized in small-scale laboratories: Computer Science and Engineering Laboratory, Microelectronics Laboratory, Telecommunications and Remote Sensing Laboratory, Microwave Laboratory and Mathematical Engineering Laboratory.
    With the evolution of technologies towards complex systems, it became clear that these laboratories could benefit from each other to tackle modern theoretical and technological challenges.
    This led to the creation of the ICTEAM institute in 2010 to gather researchers in these fields under a common organization with the mission to foster collaboration across disciplines.
  \end{block}
  \begin{block}{Making a good poster}
    See \cite{shannon1948mathematical}.
  \end{block}
\end{column}

\begin{column}{\sepwid}\end{column} % Empty spacer column

\begin{column}{\onecolwid} % The third column
  \begin{block}{Technical facilities}

    ICTEAM manages one technological platform (WELCOME) and is the reference
		institute for another platform (WINFAB):
		\begin{itemize}
			\item The \alert{WELCOME} facility (Wallonia ELectronics \& COmmunications MEasurements)
				is a state-of-the-art technology platform providing multidisciplinary tools in the
				field of electrical and electromagnetic characterization. WELCOME offers a wide
				variety of electrical and electromagnetic measurement techniques ranging from
				the physical behavior of materials, sensors and devices, to system architecture
				and signal propagation.
			\item The \alert{WINFAB} technology platform (Wallonia Infrastructure Nano FABrication) is
				dedicated to micro- and nano-fabrication. The platform maintains a 1000 m2
				cleanroom infrastructure with a large high-tech equipment fleet (more than
				50 machines), and offers access to a wide range of advanced techniques for
				thin-film deposition, thin-film etching, surface patterning, micromachining and
				back-end processes.
		\end{itemize}

		\begin{alertblock}{Key numbers}
			Our Institute comprises about 300 members including 43 academics, 180 researchers
			(PhD students and postdocs), 14 administration staff members and 18 computer
			scientists and technicians.
			About 25 PhD theses are defended yearly, and an average of 200 journal papers are
			published per year.
			The ratio between UCLouvain funding and external funding is about 12 \% UCLouvain -
			88 \% external.
		\end{alertblock}
  \end{block}
\end{column}

\end{columns} % End of all the columns in the poster

\end{frame}

\end{document}
